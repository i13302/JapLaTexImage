%全体の2段組.tex 
\documentclass[twocolumn,10pt]{jarticle} 
% \documentclass[a4paper,10pt]{jsarticle}
\usepackage{listings,jlisting}
\usepackage[dvipdfmx]{graphicx,color}
\usepackage{eclbkbox}
\usepackage{forloop}
\usepackage{url}

\setlength{\columnsep}{3zw} 
\title{\TeX とは} 
\author{筱 更治} 
\date{\today}


%%==== ソースコードのレイアウト
\lstset{%
	language=C,
	breaklines=true,%改行
	numbers=left,%
	numberstyle={\scriptsize},%
	stepnumber=1,
	numbersep=1zw,%
	lineskip=-0.5ex,%
	basicstyle=\ttfamily\footnotesize\fontsize{8}{8},
	frame=single,%
	columns=[l][l]{fullflexible},
	tabsize=4,
	xleftmargin=3zw,
	xrightmargin=3zw, % 最所先生オリジナル
	framexleftmargin=3zw, %ソースの左枠に行番号
	commentstyle={\ttfamily \color[rgb]{0,0.5,0}},
	keywordstyle={\bfseries \color[rgb]{1,0,0}},
	stringstyle={\ttfamily \color[rgb]{0,0,1}},
	literate= %特殊文字
		*{\#include}{{\textcolor[rgb]{0.7,0.3,0.5}{\#include}}}{7}
		 {\#define} {{\textcolor[rgb]{0.7,0.3,0.5}{\#define}}}{6}
		 {\#if}     {{\textcolor[rgb]{0.7,0.3,0.5}{\#if}}}{2}
		 {\#else}   {{\textcolor[rgb]{0.7,0.3,0.5}{\#else}}}{4}
		 {\#endif}  {{\textcolor[rgb]{0.7,0.3,0.5}{\#endif}}}{5}
		 {\#elif}   {{\textcolor[rgb]{0.7,0.3,0.5}{\#elif}}}{4}
		 {\#ifndef} {{\textcolor[rgb]{0.7,0.3,0.5}{\#ifndef}}}{6}
		 {\#ifdef}  {{\textcolor[rgb]{0.7,0.3,0.5}{\#ifdef}}}{5},
}


%%%%%%    TEXT START    %%%%%% 
\begin{document} 
\maketitle 
\section{\TeX について}

TeXはスタンフォード大学教授(数学)D.E.Knuth(19388~)による文書整形システムです。TeXは大抵「テフ」と読まれいます。TeXはワープロのたぐいと言えますが、より正しくは、1つのプログラム言語に近いものです。利用者によるマクロ命令によって機能を拡張することができます。今までは研究者の間でUNIX環境での稼働が一般的でしたが、今日では、個人がMacintoshOSやWindows9Xをインストールしたパーソナルコンピュータ上でTeXを動かすことが可能です。ネットワークで配布されているパッケージもありますが、最近では、安価にCD-ROMの形態で書籍に付録されているものもあり、ある程度の文法の理解は必要ですが、文書作成の種類や目的によっては、とても重宝なツールと言えます。……以下続く…… 。適当にGoogleのURL\url{https://google.com}。

\begin{lstlisting}[caption=FORMURAの定義]
#define FORMULA  0 // 途中経過を表示
\end{lstlisting}
Made time is  \pdfcreationdate .

\lstinputlisting[caption=カラツバ法を実行 bignum\_kara()]{./src/bignum_kara.txt}

\end{document}
